\documentclass[screen,acmsmall,nonacm]{acmart}

\def\FormatName#1{%
	\def\myname{KC Sivaramakrishnan}%
		\def\name{#1}%
		\ifx\name\myname
		\textbf{#1}%
		\else
		#1%
		\fi
}


\newcommand{\loud}[1]{\textbf{\textit{#1}}}
\newcommand{\R}[1]{~\\[-2mm] \noindent \textbf{#1.~~}}

\renewcommand{\shortauthors}{KC Sivaramakrishnan}

\begin{document}

\title{Research Statement}

\noindent \Large \textsc{KC Sivaramakrishnan} \hfill \textsc{Research Statement} \normalsize

\noindent \hrulefill

Over the last decade, software engineering has seen significant advancements
and shifts driven by technological innovation and changing industry demands.
Emerging applications such as internet-of-things, augmented reality and
self-driving vehicles have necessitated moving computation closer to the data
for real-time decision making, while also depending on cloud computing
platforms for AI models used to make those decisions. In order to support these
varied applications, the computing platforms have become heterogeneous, with a
mix of CPUs, GPUs and FPGAs, both on the server and the client sides. The
increasing complexity and ubiquity of software systems have also led to new
security and privacy challenges.

As software continues to eat the
world\footnote{\url{https://a16z.com/why-software-is-eating-the-world/}}, a
growing population of software developers are faced with the challenge of
building correct, secure and scalable software systems that can run on a wide
range of platforms. They must ensure correct application behaviour the in face
of asynchrony and partial failures, ensure absence of security and privacy
issues arising anywhere from programming errors to malicious attacks, all the
while providing good scalability as well as minimizing the user's perception of
latency. This remains an uphill task with current programming language
technology. No wonder then that bugs and exploits evade the programmer during
software development and testing, only to appear in production environments
with devastating consequences.

My research goal is to \loud{develop programming language abstractions and
tools that empower developers to build secure, scalable and reliable software
systems}. I believe that mathematically rigorous functional programming is
particularly suited towards this goal. A distinguishing aspect of my research
is that I spend significant effort to make these abstractions and tools
available to practitioners and thereby bridging the gap between research and
practice.

\section{Previous work}

\subsection{Concurrency and Parallelism}

In my research career so far, I have developed concurrent and parallel
programming abstractions for widely used functional programming language
compilers including the MLton Standard ML compiler~\cite{mmpar, mmgc,
KC_MARC12, Ziarek11, Ziarek09, MMJFP}, the Glasgow Haskell
Compiler~\cite{CompSA} and OCaml~\cite{RetroEffects,RetroParallel}. A notable
aspect of these works is that they are implemented in industrial-strength
compilers, and some of them are widely used by practitioners.

During my PhD studies, I led the MultiMLton project, a parallel extension of
the MLton Standard ML compiler, targeted at future many core processors. In
MultiMLton, concurrent programs are organised as a large number of cooperative
lightweight threads, that communicate by passing messages between each other. I
developed a novel asynchronous communication abstraction and a
mostly-concurrent garbage collector~\cite{MMJFP} that allowed MultiMLton to
scale to the 864-core Azul Vega3 machine. MultiMLton's multicore garbage
collector (GC) was designed to minimise inter-core communication. The key
innovation was to trade some of the ample concurrency in the source language to
offset some of the GC costs~\cite{mmgc}.

MultiMLton was designed not only for traditional cache-coherent multicore
machines, but also to take advantage of exotic architectures that provided
fine-grained control over caches. I developed a port of MultiMLton to the
non-cache-coherent Intel Single-chip Cloud Computer (SCC), which preserved the
familiar shared memory parallel programming model~\cite{KC_MARC12}. This work
took advantage of MultiMLton's ability to statically distinguish mutable and
immutable data to manage them in separate cache coherence domains. This work
won the \loud{Best Paper Award at the Intel Many-core Architecture Community
(MARC) Symposium} at RWTH, Aachen, Germany. My work on MultiMLton was
recognised by Purdue University with the \loud{Maurice H Halstead award for
outstanding research in software engineering}.

After my PhD, I joined the University of Cambridge Computer Lab as an 1851
Royal Commission and Darwin College Research Fellow, where I turned my
attention to bring native support for concurrency and parallelism to the OCaml
programming language as part of the Multicore OCaml project. Despite being one
of the most popular functional programming languages, OCaml lacked support for
native concurrency and parallelism. For concurrency, we introduced \emph{effect
handlers} into the language~\cite{RetroEffects}. Effect handlers are a
mechanism for programming with user-defined effects. Operationally, effect
handlers provide a mechanism for structured programming over delimited
continuations. Effect handlers generalise mechanisms such as exceptions,
generators, async/await and lightweight threads, which are provided as
primitives by other languages. With the addition of effect handlers, we are now
able to implement these rich mechanisms as libraries with well-defined
semantics for their interactions.

For parallelism, we have rewritten major parts of the runtime system of OCaml
to be parallelism safe, and introduced a new mostly concurrent garbage
collector that can scale to 100s of cores~\cite{RetroParallel}. This is design
minimises stop-the-world phase where all the cores are stopped running OCaml
code and the garbage collector runs. This is garbage collector is particularly
suited for latency-sensitive programs that OCaml is often used for such as
trading systems, user interfaces, and network-facing micro-services. Programs
running on multicore processors also exhibit non-trivial behaviours due to
reordering by modern multi-core hardware and compiler optimisations. We have
developed a novel \emph{relaxed memory model} for OCaml that offers local
reasoning about program fragments unlike the global reasoning required by Java
and C11 memory models~\cite{LDRF}. This memory model has been implemented in
the OCaml compiler for all the supported multicore architectures including x86,
ARM, PowerPC and RISC-V.

A major challenge with introducing concurrency and parallelism to a widely used
programming language is the existence millions of lines of legacy code, most of
which may remain sequential forever. We face the challenge of maintaining
backwards compatibility--not just in terms of the language features but also
the performance of single-threaded code. We have succeeded in achieving this
goal, and Multicore OCaml project has been merged into the mainstream OCaml
compiler and released as part of OCaml
5\footnote{\url{https://github.com/ocaml/ocaml/pull/10831}}. This makes OCaml
the first industrial-strength language to support effect handlers. The work on
Multicore OCaml has been recognised with the \loud{2023 SIGPLAN Programming
Languages Software Award} and a \loud{distinguished paper award at ICFP 2020}
for the paper that describes the GC design~\cite{RetroParallel}. I am one of
the core maintainers of OCaml, and continue to contribute to the development
and maintenance of the concurrency and parallelism features.

Multicore OCaml has also had an enormous impact on the wider community. React,
the most widely used JavaScript UI framework in the world introduced a major
feature called \emph{React Hooks}, which is directly inspired from Multicore
OCaml\footnote{\url{https://legacy.reactjs.org/docs/hooks-faq.html\#what-is-the-prior-art-for-hooks}}.
WebAssembly (Wasm) is a type-safe, efficient language for the web, introduced
as an alternative to JavaScript. All major browsers now support Wasm. Wasm is
introducing effect handlers for concurrency based on the OCaml design
(WasmFX)~\cite{WasmFX}. WasmFX brings the benefit of effect handlers to every
language that can compile to Wasm. I continue to participate in the Wasm
community group to advance the WasmFX proposal.

\subsection{Distribution}

As a natural extension of concurrent and parallel programming, I am fascinated
with the challenges with distributed programming. Unlike parallel programming,
in loosely-coupled asynchronous distributed systems (LADS), synchronisation
leads to unavailability, and hence, is avoided when possible. This makes
programming LADS challenging. Unfortunately, this has led to centralisation of
Internet services, where a few large corporations provide services to billions
of users. The personal data owned by these large corporations is a major
security and privacy concern. In order to build a resilient and decentralised
Internet economy, distributed programming needs to be made easier to enable
building \emph{local-first
software}\footnote{\url{https://www.inkandswitch.com/local-first/}}.

My research in this area has been focussed on correctly building software that
works under weaker consistency guarantees in LADS. To alleviate the burden of
developing correct programs, I developed Quelea~\cite{quelea}, a programming
model that associates user program with declarative contracts for their
consistency expectation. Quelea utilises automated theorem proving tools to
automatically and correctly insert the necessary coordination to ensure that
application's consistency expectations are preserved.

In order to reason about state changes in LADS, functional programming
principles of immutability and persistence turn out to be particularly useful.
Based on this observation, I have developed mergeable replicated data types
(MRDTs)~\cite{mrdt}. MRDTs are based on the principles of distributed version
control systems such as Git where the causal execution history is captured
through branches and merges. A key aspect of MRDTs is that the distribution
aspects of the data type are separated from their sequential behaviour, leading
to simpler concurrent reasoning and efficient implementations. The immutability
and persistence also enables verified, correct-by-construction MRDTs
(Peepul)~\cite{peepul}. The MRDTs are implemented and verified using F*, a
proof-oriented programming language. The verified MRDT implementations are
extracted to OCaml and are compatible with Irmin, a Git-like distributed
database.

\section{Current work}

Since moving to IIT Madras and then subsequently leading the technology team at
Tarides\footnote{\url{https://tarides.com}}, I have continued the work towards
my goal of empowering developers to develop correct, secure and scalable
software, now with a team of research scholars, colleagues at IIT Madras and
programming language experts and engineers at Tarides.

\paragraph{\bf Automated verification of MRDTs.} One of the downsides of Peepul
work on correct-by-construction MRDTs is that developer needs to write down a
specification for each MRDT along with a simulation relation that connects the
specification with the implementation. The verification process
machine-assisted but is a manual one. This is tedious and error-prone. As a
follow up to the Peepul work, we asked whether we can automate the proof of
correctness of MRDTs. We have managed to do this by couching the correctness
argument in \emph{linearizability}, a well-understood concurrent programming
correctness criterion. Our definition of linearizability ensures both eventual
consistency and full functional correctness, while also allowing a simple
specification framework for conflict resolution in MRDTs. We have successfully
applied our approach on a number of complex MRDT implementations.

\paragraph{\bf Securing the foundations of Unikernels.} It is well-understood
these days that memory-unsafe languages such as C and C++ lead to majority of
security vulnerabilities in widely-used
software\footnote{\url{https://www.cisa.gov/news-events/news/urgent-need-memory-safety-software-products}}.
This has prompted a major push towards memory-safe languages such as Rust and
OCaml. However, the large legacy code base in unsafe C and C++ will continue to
remain even as new code gets written in memory-safe languages. How can we allow
mixed safe and unsafe language codebases to be secure, especially when unsafe
languages can violate safe language guarantees when combined together in the
same application? To this end, we are developing FIDES, a secure extension of
the Shakti RISC-V processor that prevents security exploits in unsafe code
through hardware guards while permitting safe code to run as fast as possible.
FIDES also supports inter-process compartments to enforce isolation between
different parts of the application. The secure processor is designed to run
MirageOS, an OCaml-based library operating system for constructing Unikernels.
The aim is to target critical embedded system applications such as remote
voting
machines\footnote{\url{https://pib.gov.in/PressReleasePage.aspx?PRID=1887248}}
and point-of-sale terminals. In general, I see enormous potential for
open-source secure hardware such as Shakti processors to complement the
open-source secure software such as MirageOS to help build trustworthy systems.

\paragraph{\bf Mechanically verified garbage collectors.} In my experience
building multiple concurrent garbage collectors (GC), debugging rare and
non-deterministic failures took up significant amount of development time.
Garbage collectors are typically written in memory-unsafe languages where there
is explicit control over low-level memory management facilities. However, bugs
in the GCs can lead to security vulnerabilities in the safe language code. Can
we build practical, correct-by-construction, mechanically verified GCs? To this
end, we have built a correct-by-construction stop-the-world mark-and-sweep GC
in the F* programming language and extracted that to C using the KaRaMel
compiler. The GC includes enough features that it can be used as a replacement
for the GC in the OCaml programming language and does not compromise on the
performance. The GC and its proofs of correctness are designed in a modular
fashion, and we plan to use this modularity to extend the GC to support
incremental, generational and concurrent garbage collection.

All of the research listed above is being conducted in collaboration with IIT
Madras colleagues and PhD students, and the corresponding papers are under
submission.

\section{Future work}

In the future, I plan to continue to push towards my goal of empowering
developers to build correct, scalable and secure software. I see a number of
recent developments that will accelerate the push towards this goal.

\paragraph{\bf Securing the foundations with OCaml Oxide.} Oxidising OCaml
project~\cite{oxidising} brings Rust-style ownership and borrowing to OCaml. In
particular, it brings in the ability to reason about the lifetimes of objects
statically, which turns out to be a useful building block for a variety of
features. I plan to use the Oxidising OCaml project bring effect safety to
effect handlers in OCaml 5, stricter data sharing policies in FIDES
compartments and build better concurrent data structures that can take
advantage of static knowledge of object sharing between different threads.

\paragraph{\bf Trustworthy CodeLLMs.} CodeLLMs are specialized language models
designed and trained to understand, generate, and reason about programming
code. A major challenge in codeLLMs is to ensure that the generated code
satisfies the programmer's intent. This is particularly challenging as the
intent keeps evolving as the functional requirements of the software changes.
How can we ensure that the generated code is correct? How can we ensure that
the generated code can be evolved? I believe that lightweight formal method
techniques such as static type systems, property-based testing and
example-drive development can be profitably used to capture programmer intent
and to derive the specifications. Once we have the specifications, codeLLMs can
themselves be used to generate the correctness proofs to be machine checked in
a proof assistant such as F* or Coq. Today codeLLMs do quite well on
high-resource languages such as Python and JavaScript. If the current trend in
codeLLMs continues, programming languages will cease to be the dominant
interface for programming. I conjecture that, at this point, mathematically
rigorous functional programming languages will be preferred target language for
codeLLMs as they are today the preferred target for proof assistants.

The ideas from functional programming languages are becoming mainstream as they
are incorporated into mainstream languages. I am fortunate and excited to be at
the forefront, driving some of these changes, as we empower developers to build
correct, secure, and scalable software.

\newpage

\bibliographystyle{ACM-Reference-Format}
\bibliography{all}

\end{document}
